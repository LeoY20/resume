\documentclass[10pt, letterpaper]{article}

% Packages:
\usepackage[
    ignoreheadfoot, % set margins without considering header and footer
    top=2 cm, % seperation between body and page edge from the top
    bottom=2 cm, % seperation between body and page edge from the bottom
    left=2 cm, % seperation between body and page edge from the left
    right=2 cm, % seperation between body and page edge from the right
    footskip=1.0 cm, % seperation between body and footer
    % showframe % for debugging 
]{geometry} % for adjusting page geometry
\usepackage{titlesec} % for customizing section titles
\usepackage{tabularx} % for making tables with fixed width columns
\usepackage{array} % tabularx requires this
\usepackage[dvipsnames]{xcolor} % for coloring text
\definecolor{primaryColor}{RGB}{0, 79, 144} % define primary color
\usepackage{enumitem} % for customizing lists
\usepackage{fontawesome5} % for using icons
\usepackage{amsmath} % for math
\usepackage[
    pdftitle={Leo Yao's Resumé},
    pdfauthor={Roger Fan},
    pdfcreator={LaTeX},
    colorlinks=true,
    urlcolor=primaryColor
]{hyperref} % for links, metadata and bookmarks
\usepackage[pscoord]{eso-pic} % for floating text on the page
\usepackage{calc} % for calculating lengths
\usepackage{bookmark} % for bookmarks
\usepackage{lastpage} % for getting the total number of pages
\usepackage{changepage} % for one column entries (adjustwidth environment)
\usepackage{paracol} % for two and three column entries
\usepackage{ifthen} % for conditional statements
\usepackage{needspace} % for avoiding page brake right after the section title
\usepackage{iftex} % check if engine is pdflatex, xetex or luatex

% Ensure that generate pdf is machine readable/ATS parsable:
\ifPDFTeX
    \input{glyphtounicode}
    \pdfgentounicode=1
    % \usepackage[T1]{fontenc} % this breaks sb2nov
    \usepackage[utf8]{inputenc}
    \usepackage{lmodern}
\fi



% Some settings:
\AtBeginEnvironment{adjustwidth}{\partopsep0pt} % remove space before adjustwidth environment
\pagestyle{empty} % no header or footer
\setcounter{secnumdepth}{0} % no section numbering
\setlength{\parindent}{0pt} % no indentation
\setlength{\topskip}{0pt} % no top skip
\setlength{\columnsep}{0cm} % set column seperation
\makeatletter
\let\ps@customFooterStyle\ps@plain % Copy the plain style to customFooterStyle
\patchcmd{\ps@customFooterStyle}{\thepage}{
    \color{gray}\textit{\small Leo Yao - Page \thepage{} of \pageref*{LastPage}}
}{}{} % replace number by desired string
\makeatother
\pagestyle{customFooterStyle}

\titleformat{\section}{\needspace{4\baselineskip}\bfseries\large}{}{0pt}{}[\vspace{1pt}\titlerule]

\titlespacing{\section}{
    % left space:
    -1pt
}{
    % top space:
    0.3 cm
}{
    % bottom space:
    0.2 cm
} % section title spacing

\renewcommand\labelitemi{$\circ$} % custom bullet points
\newenvironment{highlights}{
    \begin{itemize}[
        topsep=0.10 cm,
        parsep=0.10 cm,
        partopsep=0pt,
        itemsep=0pt,
        leftmargin=0.4 cm + 10pt
    ]
}{
    \end{itemize}
} % new environment for highlights

\newenvironment{highlightsforbulletentries}{
    \begin{itemize}[
        topsep=0.10 cm,
        parsep=0.10 cm,
        partopsep=0pt,
        itemsep=0pt,
        leftmargin=10pt
    ]
}{
    \end{itemize}
} % new environment for highlights for bullet entries


\newenvironment{onecolentry}{
    \begin{adjustwidth}{
        0.2 cm + 0.00001 cm
    }{
        0.2 cm + 0.00001 cm
    }
}{
    \end{adjustwidth}
} % new environment for one column entries

\newenvironment{twocolentry}[2][]{
    \onecolentry
    \def\secondColumn{#2}
    \setcolumnwidth{\fill, 4.5 cm}
    \begin{paracol}{2}
}{
    \switchcolumn \raggedleft \secondColumn
    \end{paracol}
    \endonecolentry
} % new environment for two column entries

\newenvironment{header}{
    \setlength{\topsep}{0pt}\par\kern\topsep\centering\linespread{1.5}
}{
    \par\kern\topsep
} % new environment for the header

% \newcommand{\placelastupdatedtext}{% \placetextbox{<horizontal pos>}{<vertical pos>}{<stuff>}
%   \AddToShipoutPictureFG*{% Add <stuff> to current page foreground
%     \put(
%         \LenToUnit{\paperwidth-2 cm-0.2 cm+0.05cm},
%         \LenToUnit{\paperheight-1.0 cm}
%     ){\vtop{{\null}\makebox[0pt][c]{
%         \small\color{gray}\textit{Last updated in September 2024}\hspace{\widthof{Last updated in September 2024}}
%     }}}%
%   }%
% }%

% save the original href command in a new command:
\let\hrefWithoutArrow\href

% new command for external links:
\renewcommand{\href}[2]{\hrefWithoutArrow{#1}{\ifthenelse{\equal{#2}{}}{ }{#2 }\raisebox{.15ex}{\footnotesize \faExternalLink*}}}


\begin{document}
    \newcommand{\AND}{\unskip
        \cleaders\copy\ANDbox\hskip\wd\ANDbox
        \ignorespaces
    }
    \newsavebox\ANDbox
    \sbox\ANDbox{}

    % \placelastupdatedtext
    \begin{header}
        \textbf{\fontsize{24 pt}{24 pt}\selectfont Leo Yao}

        \vspace{0.3 cm}

        \normalsize
        \mbox{{\color{black}\footnotesize\faMapMarker*}\hspace*{0.13cm}Palo Alto, CA}%
        \kern 0.25 cm%
        \AND%
        \kern 0.25 cm%
        \mbox{\hrefWithoutArrow{mailto:youremail@yourdomain.com}{\color{black}{\footnotesize\faEnvelope[regular]}\hspace*{0.13cm}leoyao@cmu.edu}}%
        \kern 0.25 cm%
        \AND%
        \kern 0.25 cm%
        \mbox{\hrefWithoutArrow{tel:+1650-285-0101}{\color{black}{\footnotesize\faPhone*}\hspace*{0.13cm}+1 (650) 285-0101}}%
        \kern 0.25 cm%
        \AND%
        % \kern 0.25 cm%
        % \mbox{\hrefWithoutArrow{https://yourwebsite.com/}{\color{black}{\footnotesize\faLink}\hspace*{0.13cm}yourwebsite.com}}%
        % \kern 0.25 cm%
        % \AND%
        % \kern 0.25 cm%
        % \mbox{\hrefWithoutArrow{https://linkedin.com/in/yourusername}{\color{black}{\footnotesize\faLinkedinIn}\hspace*{0.13cm}yourusername}}%
        % \kern 0.25 cm%
        % \AND%
        \kern 0.25 cm%
        \mbox{\hrefWithoutArrow{https://github.com/LeoY20}{\color{black}{\footnotesize\faGithub}\hspace*{0.13cm}LeoY20}}%
    \end{header}

    \vspace{0.3 cm - 0.3 cm}

    % \section{Welcome to RenderCV!}

    %     \begin{onecolentry}
    %         \href{https://rendercv.com}{RenderCV} is a LaTeX-based CV/resume version-control and maintenance app. It allows you to create a high-quality CV or resume as a PDF file from a YAML file, with \textbf{Markdown syntax support} and \textbf{complete control over the LaTeX code}.
    %     \end{onecolentry}

    %     \vspace{0.2 cm}

    %     \begin{onecolentry}
    %         The boilerplate content was inspired by \href{https://github.com/dnl-blkv/mcdowell-cv}{Gayle McDowell}.
    %     \end{onecolentry}
    
    % \section{Quick Guide}

    % \begin{onecolentry}
    %     \begin{highlightsforbulletentries}


    %     \item Each section title is arbitrary and each section contains a list of entries.

    %     \item There are 7 unique entry types: \textit{BulletEntry}, \textit{TextEntry}, \textit{EducationEntry}, \textit{ExperienceEntry}, \textit{NormalEntry}, \textit{PublicationEntry}, and \textit{OneLineEntry}.

    %     \item Select a section title, pick an entry type, and start writing your section!

    %     \item \href{https://docs.rendercv.com/user_guide/}{Here}, you can find a comprehensive user guide for RenderCV.

    %     \end{highlightsforbulletentries}
    % \end{onecolentry}

    \section{Education}

        \begin{twocolentry}{
        % two-col takes in 1 arg for the right one and dynamic left.
        \text{August 2024 – Present}}
            \textbf{Carnegie Mellon University}
        \end{twocolentry}
        % degree, dates
        \begin{twocolentry}{
            \textit{Pittsburgh, PA}}
            \textit{B.S. in Statistics \& Machine Learning and Computer Science}
        \end{twocolentry}
        
        \vspace{0.10 cm}
        \begin{onecolentry}
            \begin{highlights}
                \item \textbf{QPA:} 3.64/4.00 
                \item \textbf{Relevant Coursework:} Principles of Imperative Computation, Introduction to Computer Systems, Principles of Functional Programming, Introduction to Computer Security, Matrices and Linear Transformations, Calculus in 3D, Probability and Statistical Inference I, Statistical Graphics and Visualization.
                % \item \textbf{Awards:} Dean's List, High Honors (Spring 2025)
            \end{highlights}
        \end{onecolentry}
    
    \section{Experience}
        \begin{twocolentry}{
            \text{September 2025 - Present}}
            \textbf{Research Assistant}
        \end{twocolentry}
        \begin{twocolentry}{
            \textit{Pittsburgh, PA}}
            \textit{FastML Lab}
        \end{twocolentry}

        \vspace{0.10 cm}
            \begin{highlights}
                \item Will conduct research on optimizing deployment of Graph Neural Networks on FPGA chips. 
                \item Will contribute to development of \href{https://fastmachinelearning.org/hls4ml/intro/introduction.html}{hls4ml}, a package translating PyTorch and Keras models to high level synthesis code for FPGA chips.
            \end{highlights}
        \begin{twocolentry}{
        \text{June 2025 – August 2025}}
            \textbf{Machine Learning Researcher}
        \end{twocolentry}
        \begin{twocolentry}{
            \textit{Pittsburgh, PA}}
        \textit{CMU SPICE Lab}
        \end{twocolentry}

        \vspace{0.10 cm}
        \begin{onecolentry}
            \begin{highlights}
                \item Cleaned and preprocessed household energy use data with R Tidyverse, Pandas, NumPy, and SciKit-Learn.
                \item Developed an artificial neural network (ANN) in PyTorch to predict households' annual cooling energy, engineering a performance improvement of over 75\% by implementing LightGBM-based feature selection and advanced hyperparameter tuning (e.g. learning rate annealing, early stopping).
                \item Created model performance graphics with MatPlotLib; presented graphics in a poster to peers.
                \item Managed codebase and tracked experiments using Git. Poster and code available upon request.
            \end{highlights}
        \end{onecolentry}

        % \vspace{0.2 cm}
        
        % \begin{twocolentry}{
        % \text{July 2023 - August 2023}}
        %     \textbf{Data Science Researcher}
        % \end{twocolentry}
        % \begin{twocolentry}{
        %     \textit{Irvine, CA}}
        %     \textit{UC Irvine COSMOS Program}
        % \end{twocolentry}

        % \vspace{0.10 cm}
        % \begin{onecolentry}
        %     \begin{highlights}
        %         \item Implemented and benchmarked several regression models (Linear, LASSO, Random Forest, etc.) to evaluate accuracy and performance in predicting brain grey matter from measured human physical features.
        %         \item Plotted results from each model with ggplot2, summarizing results in a project poster; presented poster at a research symposium to peers in the program and UCI faculty.  Poster and code available upon request.
        %     \end{highlights}
        % \end{onecolentry}

    \section{Leadership}
        \begin{twocolentry}{
            \text{November 2024 – Present}}
            \textbf{Joint Funding Committee Member}
        \end{twocolentry}
        \begin{twocolentry}{
            \textit{Pittsburgh, PA}}
            \textit{CMU Student Government}
        \end{twocolentry}

        \vspace{0.10 cm}
        \begin{onecolentry}
            \begin{highlights}
                \item Managing the distribution and allocation of approximately \$2.1 million to 300+ student organizations.
                \item Acting as a liaison between student organizations and Student Government, advocating for their financial needs during weekly JFC meetings.
            \end{highlights}
        \end{onecolentry}
    \section{Awards}
        \begin{onecolentry}
            \textbf{Dean's List, High Honors} (Spring 2025)
        \end{onecolentry}
    \section{Projects}
        \begin{twocolentry}{
        \textit{May 2025 – July 2025}}
            \textbf{Computer Systems (in C language)}
        \end{twocolentry}

        \vspace{0.10 cm}
        \begin{onecolentry}
            \begin{highlights}
                \item Developed a dynamic memory allocator in C for the Linux Platform. Achieved 74.3\% memory utilization and a throughput of 15885 KOPS (kilo-operations per second), ranking among the top of the class.
                \item Built a tiny Linux shell with job control and I/O redirection.
                \item Created a multithreaded web proxy server that facilitates communication between clients and web servers.
            \end{highlights}
        \end{onecolentry}
    \section{Skills}
        \begin{onecolentry}
            \textbf{Programming Languages:} Python, R, C, Java, Assembly \newline
            \textbf{Frameworks/Libraries:} Pandas, NumPy, SciKit-Learn, PyTorch, MatPlotLib, Tidyverse, ggplot2 \newline
            \textbf{Developer Tools:} Git, Linux

        \end{onecolentry}


    \vspace{0.20 cm}

    

    

\end{document}